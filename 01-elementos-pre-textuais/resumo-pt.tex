% -----------------------------------------------------------------------------
% Resumo
% -----------------------------------------------------------------------------

\begin{resumo}
   
   A crescente demanda por transparência levou os governos a disponibilizarem, na Internet, 
   dados que são de interesse da sociedade, são os chamados dados governamentais abertos. No 
   entanto, para as pessoas interessadas, o acesso a essas bases não é suficiente para fazer 
   uso das mesmas, a falta de conhecimento técnico pode ser um empecilho. Isso ocorre pois 
   esses dados são heterogêneos, disponíveis em diversos formatos, em grande volume e nem 
   sempre de fácil entendimento para as pessoas interessadas. Essas características dificultam 
   a integração desses dados, o que limita a capacidade de manipulação, combinação e análise 
   dos mesmos. Um dos desafios gerados por esse contexto é referente a demanda por uma 
   infraestrutura capaz de processar e integrar esses dados, viabilizando a exploração e 
   análise dessas bases. Motivado por este cenário, este trabalho propõe o WikiOlapBase,
   uma ferramenta colaborativa para a integração de dados abertos, que viabiliza a análise, 
   cruzamento e visualização desse tipo de dado. Para alcançar esse objetivo foi realizada 
   uma revisão de abordagens para processamento, armazenamento e integração de dados. 
   Esse levantamento identificou plataformas semelhantes, além de diferentes técnicas e 
   tecnologias que viabilizam a criação desse tipo de ferramenta. A partir dessa revisão foram 
   definidos os requisitos e a arquitetura do WikiOlapBase. Com essas decisões tomadas a 
   ferramenta foi implementada. Posteriormente foi feita a avaliação de usabilidade da 
   plataforma, para avaliar sua adequação ao uso sob a perspectiva dos usuários. 
   Os resultados mostram a aceitação da ferramenta por parte dos usuários, bem como sua 
   adequação ao uso.

    \textbf{Palavras-chave}: Integração de dados. Dados abertos. Big Data. Software colaborativo.
\end{resumo}

% -----------------------------------------------------------------------------
% Escolha de 3 a 6 palavras ou termos que descrevam bem o seu trabalho. As palavras-chaves são utilizadas para indexação.
% A letra inicial de cada palavra deve estar em maiúsculas. As palavras-chave são separadas por ponto.
% -----------------------------------------------------------------------------
