% -----------------------------------------------------------------------------
% Conclusão
% -----------------------------------------------------------------------------

\chapter{Conclusões e Trabalhos Futuros}
\label{chap:conclusao}

Este trabalho foi desenvolvido com o objetivo de criar o WikiOlapBase, uma ferramenta
colaborativa que permite a integração de dados abertos. A metodologia empregada para este 
projeto consistiu nas etapas de levantamento de ferramentas similares existentes na literatura, 
levantamento dos requisitos do sistema, elaboração de uma arquitetura que satisfizesse os
requisitos, desenvolvimento e teste da ferramenta e avaliação da mesma sob a perspectiva dos
usuários. 

Os resultados encontrados, a partir do Teste de Usabilidade, demonstram que o WOB é uma 
ferramenta útil, satisfatória e adequada ao uso, que permite a integração de dados abertos de 
forma colaborativa. Além disso, o levantamento feito das ferramentas similares, presente neste 
trabalho, também é relevante, pois aborda uma análise comparativa, que permite explorar 
diferentes abordagens e técnicas, para a criação de ferramentas para integração de dados.

Assim, este trabalho apresenta contribuições tanto práticas quanto científicas. Como contribuição
prática, a ferramenta permitiu adicionar elementos de colaboração no processo de integração de 
dados, algo que ainda era limitado em outras ferramentas similares. Além disso, outra contribuição
prática, é a disponibilização do código fonte\footnote{https://github.com/pedromb/wikiolapbase} gerado.
Em termos científicos este trabalho delimitou as vantagens e desvantagens de diferentes 
abordagens para o processamento, integração e armazenamento de dados, contribuindo para o 
avanço na discussão destes temas.

\section{Trabalhos Futuros}
\label{sec:trablhos}

Como trabalho futuro é proposto a evolução da ferramenta, uma lista de melhorias foi gerada
e pode ser acessada no Apêndice \ref{apendiceB}. Além disso outros tipos de teste podem, e 
devem ser realizados para avaliar o desempenho da ferramenta, como por exemplo a geração de 
um \textit{benchmarking} para comparar a abordagem proposta com outras.

Também vale ressaltar que o projeto já prevê uma segunda fase, na qual deve ser desenvolvida 
uma ferramenta de visualização de dados, que consome a API REST disponibilizada neste trabalho. 
Finalmente, devido a natureza distribuída da maioria das tecnologias utilizadas no 
desenvolvimento dessa ferramenta, a infraestrutura necessária, e a melhor maneira de 
disponibilizá-la para o público, também deve ser estudada.


% -----------------------------------------------------------------------------
% OBS: a norma ABNT estabelece que em qualquer tipo de trabalho acadêmico monográfico
% deve haver um capítulo de conclusão
% -----------------------------------------------------------------------------
