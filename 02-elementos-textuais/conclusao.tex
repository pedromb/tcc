% -----------------------------------------------------------------------------
% Conclusão
% -----------------------------------------------------------------------------

\chapter{Conclusões e Trabalhos Futuros}
\label{chap:conclusao}

Este trabalho foi desenvolvido com o objetivo de criar uma ferramenta colaborativa que permitisse
a integração de dados abertos. Dessa forma, a ferramenta desenvolvida deveria viabilizar o
cruzamento, análise e visualização dos dados integrados. Para isso foram realizadas duas etapas,
a primeira consistiu na revisão de abordagens para processamento, armazenamento e integração
de dados. A partir dessa revisão também foram levantadas ferramentas que possuíam objetivos 
semelhantes com os propostos por este trabalho. A segunda etapa consistiu na definição de 
requisitos para a ferramenta proposta, bem como sua arquitetura e sua implementação.

O WikiOlapBase, ferramenta implementada ao longo desse trabalho permite a integração de dados
abertos de forma colaborativa. Além disso, ela possibilita o acesso a base integrada por 
meio de uma API REST, viabilizando assim que qualquer usuário faça análises, cruze
dados e gere visualizações a partir do repositório disponível. Essas conclusões são corroboradas
pela avaliação de usabilidade que foi apresentada no capítulo \ref{chap:avaliacao}, no qual 
foi demonstrada a adequação ao uso da ferramenta bem como sua aceitação por parte dos usuários.
Além disso, no capítulo \ref{chap:desenvolvimento} foi possível notar que o grande diferencial da
ferramenta em relação às outras é seu aspecto colaborativo e a capacidade de relacionar dados
que estejam disponíveis no repositório.

Essa ferramenta possui contribuições tanto práticas quanto científicas. Em termos práticos essa 
contribuição se deu devido a revisão e implementação de soluções inovadoras para processamento,
armazenamento e integração de dados. Isso pode ser verificado tanto neste trabalho quanto no
código fonte \footnote{https://github.com/pedromb/wikiolapbase} da ferramenta, que já está disponível para o público. Em termos científicos este
trabalho delimitou as vantagens e desvantagens de diferentes abordagens para o processamento, 
integração e armazenamento de dados, contribuindo para o avanço na discussão destes temas.

\section{Trabalhos Futuros}
\label{sec:trablhos}

Como trabalho futuro é proposto a evolução da ferramenta, uma lista de melhorias foi gerada
e pode ser acessada no Apêndice \ref{apendiceB}. Além disso outros tipos de teste podem e 
devem ser realizados para avaliar o desempenho da ferramenta, como por exemplo a geração de 
um \textit{benchmarking} para comparar a abordagem proposta com outras. Também vale ressaltar
que o projeto já prevê uma segunda fase na qual deve ser desenvolvida uma ferramenta de 
visualização de dados que consome a API REST disponibilizada neste trabalho. Finalmente, devido a 
natureza distribuída da maioria das tecnologias utilizadas no desenvolvimento dessa ferramenta, 
a infraestrutura necessária e a melhor maneira de disponibilizá-la para o público também deve 
ser estudada.


% -----------------------------------------------------------------------------
% OBS: a norma ABNT estabelece que em qualquer tipo de trabalho acadêmico monográfico
% deve haver um capítulo de conclusão
% -----------------------------------------------------------------------------
