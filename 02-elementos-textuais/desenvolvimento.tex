\chapter{Desenvolvimento}
\label{chap:desenvolvimento}

Neste capítulo serão apresentadas todas as etapas referentes ao desenvolvimento da ferramenta
WikiOlapBase, que permite a integração de dados abertos de forma colaborativa. Na seção
\ref{sec:requisitos} serão explicitados os requisitos levantados para a ferramenta. 
Posteriormente, na seção \ref{sec:arquitetura} será mostrada a arquitetura proposta para o 
WOB. Finalmente na seção \ref{sec:wob} a ferramenta será apresentada, ainda nessa seção será 
feita uma análise comparativa entre o WOB e as ferramentas encontradas na literatura.

\section{Levantamento de Requisitos}
\label{sec:requisitos}

Essa etapa busca definir as funcionalidades e características do software proposto. Esse 
levantamento ocorreu a partir da revisão da literatura e através de uma reunião de 
\textit{brainstorming} no dia 29 de abril de 2016 com três especialistas que possuem mais 
de oito anos de experiência na área de processamento e análise de dados. O resultado gerado 
foi a lista de requisitos mostrada no Quadro \ref{quadro:requisitos}.

\begin{quadro}[!htb]
    \centering
    \caption{Requisitos do WOB}
    \label{quadro:requisitos}
    \begin{tabular}{|p{2cm}|p{13cm}|}
        \hline
        Identificador   &   Requisito \\
        \hline
        RF\_1    &  A ferramenta deve permitir a importação de dados, de forma a manter o significado dos dados originais \\   
        \hline
        RF\_2    &  A ferramenta deve ser capaz de converter diferentes formatos para o modelo de dados definido. \\        
        \hline
        RF\_3    &  A ferramenta deve permitir aos usuários o acesso aos dados presentes no banco de dados integrado da ferramenta. \\
        \hline
        RF\_4    &  A ferramenta deve permitir a definição de metadados que se relacionam com um determinado conjunto de dados. \\
        \hline
        RF\_5    &  A ferramenta deve ser capaz de estabelecer relacionamento entre conjunto de dados diferentes. \\
        \hline
        RF\_6    &  A ferramenta deve aceitar arquivos compactados \\
        \hline
        RF\_7    &  A ferramenta deve possibilitar a divisão dos conjuntos de dados em múltiplos arquivos para envio. \\        
        \hline
        RF\_8    &  A ferramenta deve disponibilizar uma interface para que outras aplicações acessem os dados presentes na base de dados integrada. \\
        \hline
        RNF\_1    &  A ferramenta deve ser capaz de armazenar dados em larga escala \\
        \hline
        RNF\_2    &  A ferramenta deve otimizar o tempo de consulta aos dados. \\ 
        \hline   
    \end{tabular}
    \fonte{O Autor}
\end{quadro}

A partir dos requisitos levantados foi possível escolher as tecnologias que permitiriam atender
tais requisitos, bem como propor uma arquitetura para a ferramenta. Na seção a seguir essas 
escolhas são expostas e justificadas.

\section{Arquitetura do WikiOlapBase}
\label{sec:arquitetura}

\section{WikiOlapBase}
\label{sec:wob}