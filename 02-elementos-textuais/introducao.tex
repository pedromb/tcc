% -----------------------------------------------------------------------------
% Introdução
% -----------------------------------------------------------------------------

\chapter{Introdução}
\label{chap:introducao}

Com a crescente demanda popular por mais transparência das ações governamentais, novas
políticas de publicidade dessas ações vêm sendo implementadas. Segundo \citeonline{vaz2010dados}
as tecnologias de comunicação e informação (TICs) pemitiram potencializar essa
transparência, um processo que se deu em três iniciativas, conforme descrito a seguir.

Inicialmente, os governos passaram a publicar informações de forma limitada em seus 
\textit{websites}, ou seja, decidiam o que e como seria visualizado. Em seguida, 
visando viabilizar a interação entre os usuários e bases de dados governamentais, a segunda 
iniciativa de transparência consistiu em permitir a realização de consultas para cruzamento 
e filtros dos dados, o que favoreceu o processo de análise das informações. Essas iniciativas 
eram limitadas, pois não permitiam a obtenção dos dados sem tratamentos, em seu formato original. 
Surgiu assim o conceito de dados governamentais abertos (DGA), nos quais, além de 
disponibilizar consultas e relatórios, o governo disponibiliza seus dados em estado bruto
(i.e., sem pré-processamento), o que permite sua livre manipulação, processamento e 
análise \cite{vaz2010dados}.

Em meio a esse contexto, foi criada no Brasil a Lei de Acesso à Informação 
(Lei nº 12.527/2011), que permite a qualquer cidadão a obtenção de dados e informações 
de qualquer entidade pública. Além disso, essa lei prevê a chamada “Transparência Ativa”, 
que determina que os órgãos públicos se antecipem aos pedidos e publiquem seus dados na 
Internet. Com isso, foi criado o Portal Brasileiro de Dados Abertos, no qual o 
governo federal disponibiliza dados, em estado bruto, que são de interesse público. 

No entanto, para a maioria das pessoas interessadas, a disponibilidade de acesso a essas 
bases de dados não é suficiente para fazer uso das mesmas, a falta de conhecimento técnico, 
em muitos casos, se torna um empecilho \cite{graves2013}. Isso ocorre porque 
os dados são heterogêneos, disponíveis em diversos formatos, em grande 
volume e nem sempre de fácil entendimento para as pessoas interessadas. Essas características 
dificultam a integração entre os dados, o que limita a capacidade de manipulação, 
combinação e análise dos mesmos \cite{hoxha2011open}. Ou seja, a forma como atualmente esses 
dados estão disponibilizados, não permite a obtenção de informações relevantes sem o uso de 
ferramentas computacionais que auxiliem no processamento, na visualização e análise desses 
dados \cite{vaz2010dados}. 

Esse contexto gera dois desafios: o primeiro é referente a demanda por uma infraestrutura 
capaz de processar e integrar os DGA, viabilizando a exploração e análise dessas bases. 
O segundo é referente a demanda por uma ferramenta, alimentada por essa infraestrutura, capaz 
de gerar análises e visualizações sem a necessidade de conhecimento técnico do usuário 
\cite{graves2013}.

Motivado por esses desafios, este trabalho visa propor o WikiOlapBase, uma ferramenta 
colaborativa que seja capaz de processar e integrar dados abertos. Isso significa gerar uma 
nova base de dados integrada, que seja mantida pelos usuários interessados no processamento, 
na visualização e análise desses dados. O objetivo dessa ferramenta é prover uma infraestrutura
base para outras, de modo a viabilizar a análise e visualização de grandes 
volumes de dados, mesmo por pessoas sem conhecimento técnico na área de Computação. 

Para alcançar o objetivo proposto, este trabalho foi dividido em duas fases, subdivididas 
em etapas. A primeira fase consistiu na revisão da literatura acerca de abordagens para 
processamento, armazenamento e integração de dados no cenário dos DGA, bem como na definição 
dos requisitos da ferramenta. Na segunda fase a arquitetura da ferramenta foi proposta, 
seguida do seu desenvolvimento e avaliação. A avaliação foi feita utilizando um teste de 
usabilidade. Isso foi feito com o objetivo de demonstrar a adequação ao uso da ferramenta
proposta, vale ressaltar que no escopo deste trabalho não foram realizadas avaliações de 
desempenho.

A principal contribuição prática deste trabalho é a criação de uma ferramenta colaborativa para
a integração de dados abertos, cujo código fonte está disponível ao público. Além disso, outra
contribuição, em termos práticos, é a revisão e utilização de soluções emergentes para
processamento e integração de grandes volumes de dados, um dos grandes desafios técnicos 
enfrentados pela área de \textit{Big Data} \cite{jagadish2014}. Em termos científicos, este
trabalho contribuiu no avanço de tecnologias e abordagens para processamento e integração
de dados de forma colaborativa, delimitando as vantagens e desvantagens de cada um.

Em termos de resultados, o Teste de Usabilidade conduzido demonstrou que o WikiOlapBase é uma
ferramenta colaborativa, adequada ao uso, que auxilia no processo de integração de dados 
abertos. Além disso, esse trabalho apresenta um levantamento de ferramentas similares. Isso
pode auxiliar tanto na escolha de qual ferramenta adotar, quanto na definição da arquitetura
para pré-pocessamento e integração colaborativa de dados.

Este trabalho se encontra dividido da seguinte forma: o Capítulo 
\ref{chap:fundamentacaoTeorica} apresenta conceitos e tecnologias fundamentais para o 
entendimento do trabalho; o Capítulo \ref{chap:trabRelac} explicita outras ferramentas 
existentes que viabilizam a integração, visualização e análise de dados abertos; o 
Capítulo \ref{chap:metodologia} indica a metodologia utilizada no desenvolvimento deste 
trabalho; o Capítulo \ref{chap:desenvolvimento} apresenta o desenvolvimento da ferramenta 
proposta, desde o levantamento de requisitos, passando pela arquitetura até a 
apresentação da ferramenta em si; o Capítulo \ref{chap:avaliacao} apresenta a metodologia e 
resultados gerados a partir da avaliação de usabilidade do WikiOlapBase, por fim, o Capítulo 
\ref{chap:conclusao} apresenta as conclusões geradas e trabalhos futuros propostos.