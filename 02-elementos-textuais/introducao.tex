% -----------------------------------------------------------------------------
% Introdução
% -----------------------------------------------------------------------------

\chapter{Introdução}
\label{chap:introducao}

Nos últimos anos vêm crescendo a demanda popular por mais transparência das ações 
governamentais. Em resposta à essa pressão, novas políticas vêm sendo implementadas. 
Segundo \citeonline{vaz2010dados} as tecnologias de comunicação e informação permitiram a 
potencialização da promoção de transparência, um processo que se deu em três fases. 

Inicialmente os governos passaram a publicar informações de forma limitada em seus 
\textit{websites}, ou seja, decidiam o que e como a informação seria vista. A segunda fase 
se caracteriza pela capacidade de interação entre usuário e as bases de dados governamentais. 
Era possível elaborar consultas para cruzar e filtrar dados, o que permitiu o acesso a 
informações mais detalhadas. As propostas anteriores eram limitadas pois não permitiam a 
obtenção dos dados  sem tratamentos, em seu formato original. Surgiu assim o conceito de 
dados governamentais abertos (DGA), onde além de disponibilizar consultas e relatórios, o 
governo disponibiliza seus dados em estado bruto o que permite sua livre manipulação, 
processamento e análise \cite{vaz2010dados}.

Em meio a esse contexto, foi criada no Brasil a Lei de Acesso à Informação 
(Lei nº 12.527/2011), que permite a qualquer cidadão a obtenção de dados e informações 
de qualquer entidade pública. Além disso, essa lei prevê a chamada “Transparência Ativa”, 
que determina que os órgãos públicos se antecipem aos pedidos e publiquem seus dados na 
Internet. Com isso foi criado o Portal Brasileiro de Dados Abertos, onde o governo federal 
disponibiliza dados, em estado bruto, que são de interesse público. 

No entanto, para a maioria das pessoas interessadas, a disponibilidade de acesso a essas 
bases de dados não é suficiente para fazer uso das mesmas, a falta de conhecimento técnico, 
na maioria dos casos, se torna um empecilho \cite{graves2013}. Isso ocorre porque, na 
maioria das vezes, os dados são heterogêneos, disponíveis em diversos formatos, em grande 
volume e nem sempre de fácil entendimento para as pessoas interessadas. Essas características 
dificultam a integração entre esses dados, o que limita a capacidade de manipulação, 
combinação e análise dos mesmos \cite{hoxha2011open}. Ou seja, a forma como atualmente esses 
dados estão disponibilizados não permite a obtenção de informações relevantes sem o uso de 
ferramentas computacionais que auxiliam no processamento, na visualização e análise desses 
dados \cite{vaz2010dados}. 

Esse contexto gera dois desafios: o primeiro é referente a demanda por uma infraestrutura 
capaz de processar e integrar esses dados, viabilizando a exploração e análise dessas bases. 
O segundo é referente a demanda de uma ferramenta, alimentada por essa infraestrutura, capaz 
de gerar análises e visualizações sem a necessidade de conhecimento técnico do usuário 
\cite{graves2013}.

Motivado por esses problemas, este trabalho visa propor uma ferramenta colaborativa que 
seja capaz de processar e integrar dados abertos. Isso significa gerar uma 
nova base de dados integrada que seja mantida pelos usuários interessados no processamento, 
na visualização e análise desses dados. O objetivo dessa ferramenta é servir como 
infraestrutura básica para outras, viabilizando assim, a análise e visualização de grandes 
volumes de dados mesmo por pessoas sem conhecimento técnico. 

Para alcançar o objetivo proposto, este trabalho foi dividido em duas fases, subdivididas 
em etapas. A primeira fase consistiu na revisão da literatura acerca de abordagens para 
processamento, armazenamento e integração de dados no cenário dos DGA, bem como na definição 
dos requisitos da ferramenta. Na segunda fase a arquitetura da ferramenta foi proposta, 
seguida do seu desenvolvimento e avaliação.

O principal resultado desse projeto foi a construção do WikiOlapBase (WOB), uma ferramenta 
colaborativa que permite a integração de dados abertos. A viabilização de acesso de qualidade 
para essas informações tem grande valor para diversos setores da sociedade \cite{vaz2010dados}. 
Ainda, a partir dos testes de usabilidade realizados foi possível observar que a ferramenta é 
adequada ao uso e permite a colaboração entre usuários, atingindo, dessa forma, dois de seus
principais objetivos. Além disso, esse trabalho contribuiu, em termos práticos, ao revisar e 
implementar soluções inovadoras para processamento e integração de um grande volume de dados, 
um dos grandes desafios técnicos enfrentados pela área de \textit{Big Data} 
\cite{jagadish2014}. Em termos científicos este trabalho contribuiu no avanço de 
tecnologias e abordagens para o processamento e integração de dados, delimitando as 
vantagens e desvantagens de cada um.

O resto deste trabalho se encontra dividido da seguinte forma: o Capítulo 2 apresenta
conceitos e tecnologias fundamentais para o entendimento do trabalho; o Capítulo 3 explicita
outras ferramentas existentes que viabilizam a visualização e análise de dados; o Capítulo 4
indica a metodologia utilizada no desenvolvimento deste trabalho; o Capítulo 5 apresenta o 
desenvolvimento da ferramenta proposta, desde o levantamento de requisitos, passando pela
arquitetura proposta até a apresentação da ferramenta em si; o Capítulo 8 apresenta a 
metodologia e resultados gerados a partir da avaliação de usabilidade do
WikiOlapBase, por fim, o Capítulo 9 apresenta as conclusões geradas e trabalhos futuros
propostos.