% -----------------------------------------------------------------------------
% Trabalhos Relacionados
% -----------------------------------------------------------------------------

\chapter{Trabalhos Relacionados}
\label{chap:trabRelac}

Na literatura são encontrados trabalhos referentes a arquitetura de sistemas de visualização,
integração e análise de dados, no contexto dos DGA.  Por exemplo, \citeonline{graves2013} 
mostram como o uso de visualizações podem beneficiar a população, que não possui conhecimento 
técnico, no contexto de DGA. Além disso, os autores demonstram a necessidade de criar 
mecanismos de exploração para navegar por dados e metadados, e quais as funcionalidades uma 
ferramenta deve contemplar para facilitar a criação de visualizações.

Para isso, são relatadas três fases em que problemas relacionados a visualizações aparecem, 
como tratar esses problemas, e a apresentação de um protótipo que os resolvem. A primeira fase 
é a de criação, quando ocorre o processamento dos dados que serão usados. Nessa fase, deve-se 
resolver questões sobre como tratar dados em formatos diversos, e como combinar dados de 
diferentes bases. A segunda é a fase de exploração, o maior problema nessa fase é a falta de 
informação que garante a qualidade da visualização, como a origem dos dados e o histórico de 
alterações feitas no conjunto de interesse. A última é a fase de adaptar a visualização para 
gerar outros conhecimentos. Nessa fase, devem ser tratados problemas referentes a capacidade 
de modificação dos dados implícitos na visualização, por exemplo, utilizar a média de uma 
métrica no lugar da mediana \cite{graves2013}.

O protótipo proposto por \citeonline{graves2013} utiliza os princípios de \textit{linked data}, 
para integrar os dados e torná-los legíveis tanto para humanos quanto para máquinas. Além 
disso, possui uma interface gráfica que permite a criação de visualizações.

Também seguindo os princípios de \textit{linked data}, \citeonline{hoxha2011open}, propõem a 
utilização de  tecnologias da web semântica para realizar a integração entre dados de 
diferentes organizações governamentais. Os autores propõem uma abordagem composta por três 
módulos, o primeiro é responsável por modelar uma ontologia e converter os dados não 
processados, utilizando o formato do RDF. O segundo é uma interface para consulta a essa base 
de conhecimento, composto por uma interface gráfica e mecanismo para consultas utilizando o 
\textit{Sparql Protocol and RDF Query Language} (SPARQL). O terceiro módulo é uma ferramenta 
de visualização, que faz uso dessa interface de consulta.  Finalmente, os autores sugerem uma 
implementação composta por quatro etapas: processamento dos dados, agregação da informação, 
visualização gráfica e contribuição da comunidade.

Assim como os trabalhos anteriores, \citeonline{ding2010data}, vêm trabalhando em uma 
iniciativa para integrar os dados do Data.gov\footnote{ http://www.data.gov/} (página web 
mantida pelo governo dos Estados Unidos da América, no qual dados referentes ao governo são 
disponibilizados) utilizando os princípios de \textit{linked data}. Os autores mostram como 
as tecnologias de web semântica são utilizadas para converter e integrar esses dados. Para 
isso, quatro problemas são tratados: (1) como tornar os dados capazes de fazer parte da 
nuvem do \textit{linked data}, (2) como conectar esses dados entre si e com fontes externas, 
(3) como tornar esses dados utilizáveis para usuários e desenvolvedores e (4) como preservar 
o histórico de dados \cite{ding2010data}.

A solução apresentada por \citeonline{ding2010data} começa tratando das transformações 
necessárias para adequar os dados, isso é feito através da conversão para o formato RDF. 
Após isso, os dados são enriquecidos através de processos semiautomáticos, nos quais os valores 
semânticos são associados a identificadores uniformes de recursos (URIs) que possuem 
relevância, para isso é utilizado o Semantic MediaWiki\footnote{https://www.semantic-mediawiki.org/}, 
que permite a usuários colaborem na edição de conteúdos semânticos. Esses dados são 
disponibilizados através de um \textit{webservice} SPARQL, que permite a integração dos dados 
com APIs convencionais, como a Google Visualization API. Por fim, é discutida a importância de 
metadados, que devem permitir avaliar o histórico dos dados.

Fora do contexto dos DGAs, também existem trabalhos que tratam da arquitetura de sistemas de 
visualização. \citeonline{viegas2007} discutem o desenho e desenvolvimento do ManyEyes, um 
\textit{website} no qual usuários podem enviar dados, criar visualizações interativas e 
discutir tais visualizações. O objetivo é permitir a colaboração e análise de dados de forma 
social.

Segundo os autores, as decisões de design envolvem três aspectos; (1) a visualização da 
informação, (2) a coleta de dados e manipulação por parte dos usuários, e (3) a colaboração 
de forma social. O site incentiva os usuários a disponibilizarem os metadados e oferece 
suporte para dados no formato de tabelas e texto não estruturado \cite{viegas2007}. 
Por fim, os autores discutem os tipos de visualização disponibilizados, como é feito o 
mapeamento dos dados para as visualizações, e os aspectos sociais da ferramenta.

Trabalhos, como o realizado por \citeonline{tang2004}, abordam decisões de projeto, para a 
arquitetura de sistemas de visualização de dados. Para isso, \citeonline{tang2004} discutem 
os desafios enfrentados no contexto do Rivet\footnote{https://graphics.stanford.edu/projects/rivet/}, 
um ambiente para desenvolvimento de visualizações. Inicialmente são discutidos três aspectos 
fundamentais: (1) o modelo de dados, (2) a forma de envio, ou seja, como os dados podem ser 
importados para a ferramenta, e (3) as capacidades de transformação que devem existir. 

Em relação ao modelo de dados, é discutido as vantagens e desvantagens do modelo relacional, 
comumente utilizado na implementação de sistemas. Para a forma de envio, os autores discutem 
a importância de qualquer dado importado parecer igualmente expressivo para os usuários, ou 
seja, independente do formato a informação armazenada deve ser a mesma. Para isso eles 
disponibilizam diversas formas de se importar dados, passando por conversores CSV até 
drivers para conexão com banco de dados SQL. Quando discutindo os tipos de transformação, 
os autores mostram algumas opções que parecem ser comuns no processo de análise, e, 
portanto, essenciais em um sistema de visualização, como agregação e contagem, ordenação e 
filtros \cite{tang2004}. Além desses aspectos também são tratadas questões como a 
importância de metadados, a modularização na arquitetura e pôr fim a forma de especificação 
para gerar as visualizações.

Para melhor compreender as ferramentas apresentadas, o Quadro \ref{quadro:comparativo} sintetiza as principais 
características das mesmas.

\begin{quadro}[!htb]
    \centering
    \caption{Comparação entre os sistemas encontrados na literatura}
    \label{quadro:comparativo}
    \begin{tabular}{|p{1.75cm}|p{1.75cm}|p{1.75cm}|p{1.75cm}|p{1.75cm}|p{1.75cm}|p{1.5cm}|p{1.5cm}|}
        \hline
Referência & Modelo de Dados & Forma de acesso aos dados & Formato de importação dos dados & Importa- ção de dados por usuários & Acesso a base de dados de outros usuários & Disponi- bilização de metadados & Cruza- mento entre dados\footnotemark\\
        \hline
OpenData- Vis - \citeonline{graves2013} & Linked Data & Interface gráfica & Não especificado & Não especificado & Não especificado & Sim & Não \\
        \hline          
\citeonline{hoxha2011open} & Linked Data & Interface gráfica e consultas SPARQL & Diferentes formatos (XML, CSV, texto) & Não & Não & Sim & Não \\
        \hline
Data-GovWiki - \citeonline{ding2010data} & Linked Data & Web- service SPARQL & CSV & Não & Não & Sim & Não\\
        \hline
Many Eyes - \citeonline{viegas2007} & Tabela e texto não estruturado & Interface gráfica & Texto separado por tabulação. & Sim & Sim & Sim & Não \\
        \hline
Rivet - \citeonline{tang2004} & Relacional & API REST & CSV, MDX e conexões SQL & Sim & Não & Sim & Não \\
        \hline   
    \end{tabular}
    \fonte{O Autor}
\end{quadro}
\footnotetext{Utilizando dados enviados por outros usuários} \newpage

Embora os trabalhos apresentados contemplem informações sobre a infraestrutura que suporta 
os sistemas de visualização de dados, os autores não detalham os projetos das arquiteturas. 
Também é possível notar que, embora enalteçam a importância do aspecto social na análise de 
dados governamentais abertos, não propõem alternativas concretas que permitam a colaboração 
nas fases mais elementares do processo de análise, por exemplo, durante a inserção ou o 
gerenciamento dos dados.

A ferramenta aqui proposta busca dar suporte a sistemas de visualização colaborativos, e 
estender a capacidade de colaboração para além da geração e análise de visualizações. Ou 
seja, permitir, de forma colaborativa, o processamento e integração de dados distribuídos, 
e estabelecer relacionamento entre esses dados.